\documentclass[letterpaper,11pt]{article}

\usepackage{listings}
\usepackage{color}

\definecolor{dkgreen}{rgb}{0,0.6,0}
\definecolor{gray}{rgb}{0.5,0.5,0.5}
\definecolor{mauve}{rgb}{0.58,0,0.82}

\lstset{frame=tb,
  language=Python,
  aboveskip=3mm,
  belowskip=3mm,
  showstringspaces=false,
  columns=flexible,
  basicstyle={\small\ttfamily},
  numbers=none,
  numberstyle=\tiny\color{gray},
  keywordstyle=\color{blue},
  commentstyle=\color{dkgreen},
  stringstyle=\color{mauve},
  breaklines=true,
  breakatwhitespace=true,
  tabsize=3
}

\usepackage{setspace}
\usepackage{graphicx}
\usepackage{indentfirst}
\usepackage{bm}    %for textbf
\usepackage{amsmath}
\usepackage{amsfonts}   %for mathbb
\allowdisplaybreaks[4]  %from {amsmath}
\newcommand{\independent}{\rotatebox[origin=c]{90}{$\models$}}  %from {graphicx}
\usepackage{geometry}
\geometry{letterpaper, scale=0.8}  %from {geometry}
\author{Yuan Yin A20447290}
\title{MATH 584 Homework 2}
\begin{document}\large
\maketitle
\begin{spacing}{1.2}  %from {setspace}
\section*{Problem 1}
\subsection*{(a)}
For this question, we have only the first asset and the riskless one. For each window, I firstly compute the estimation of parameters in ``Calibrition window'', which is the $N=250$ part. And then I find my optimal strategy according to the formula we got for Example 4 on class notes. Notice here the variance for riskless asset and covariance between risky asset and riskless are both zero.

Then I compute the PnL using the ``Trading window'', which is the $T=100$ part. And I save the result in file ``PnL\_in\_a.csv''. And I got the annualized mean: $1.740221402373899$, annualized variance: $0.06943642447261976$, and the Sharpe Ratio is: $6.60405727172277$.

If we want to compute the other asset with riskless, there is a variable in the code called: ``num\_asset'' which takes value $0,1,\ldots,63$ to represent different asset.

\subsection*{(b)}
For this question, I firstly substract the Volume data and compute our factor which is ``Exponential-Weighted-Average-Return''. And then I do linear regression between the factor and daily return to get coefficient for $a$ and $c$.

Then similar to the first problem, I firstly compute the estimator in each ``Calibrition window'' and then do DPP in each ``Trading window'' so that I can get a matrix of optimal strategy which represents the optimal strategy for different grid point of factor $f$ and time point $t$.

Then when computing PnL, I use interpolation to get the ``real optimal strategy'' on each time point and get the PnL process.

What we have to notice here is when doing DPP and computing PnL, the riskless return is not zero so a little bit different from the formula on lecture notes.

I save the PnL process in the file ``PnL\_in\_b.csv''. And I got the annualized mean: $39.051502045438575$, annualized variance: $34.885695320810775$, and the Sharpe Ratio is: $6.611713876937684$. We can see that althogh the mean is much higher than the previous one, the variance also increases. But luckily we get a little bit higher Sharpe Ratio.

Similarly, if we want to repeat this question with any other assets, we just need to change the ``num\_asset''.

\subsection*{(c)}
Similar to previous question, I firstly use the ``Calibrition window'' to do estimation. Then use the ``Trading window'' to do DPP and compute the optimal strategy as a matrix.

Then I do interpolation to compute the PnL with transaction cost. And I save the result in file ``PnL\_in\_c.scv''. I got the annualized mean: $0.28153354173492134$, annualized variance: $0.006363812582328052$, and the Sharpe Ratio is: $3.5291608544159025$.

Notice that the Sharpe Ratio for this one is much lower than the previous due to the consideration of transaction costs.

Similarly, if we want to repeat this question with any other assets, we just need to change the ``num\_asset''.

\end{spacing}
\end{document}